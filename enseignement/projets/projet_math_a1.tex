\documentclass[a4paper, 11pt]{article}

\usepackage[latin1]{inputenc}
\usepackage[T1]{fontenc}
\usepackage[french]{babel}
\usepackage{color}
\usepackage{listings}
\usepackage{amsmath,amssymb,amsfonts}
\usepackage{multicol}
\usepackage{epsfig}
\usepackage{graphicx}
\usepackage{algorithm}
\usepackage{algpseudocode}

%%%%%%%%%%%%%%%%%%%%%%%%%%%%%
%
% Environnement
%
%%%%%%%%%%%%%%%%%%%%%%%%%%%%%

%%%%%%%%%%%%%%%%%%%%%%%%%%%%%
%
% Commandes generales
%
%%%%%%%%%%%%%%%%%%%%%%%%%%%%%

\renewcommand{\leq}{\leqslant}
\renewcommand{\geq}{\geqslant}
\renewcommand{\(}{\left (}
\renewcommand{\)}{\right )}
\newcommand{\abs}[1]{\left|{#1}\right|}
\newcommand{\norm}[1]{||{#1}||}

\newcommand{\C}{\ensuremath{\mathbb{C}}}
\newcommand{\R}{\ensuremath{\mathbb{R}}}
\newcommand{\Z}{\ensuremath{\mathbb{Z}}}
\newcommand{\N}{\ensuremath{\mathbb{N}}}


\newcommand{\Mat}[2]{{\mathcal{M}_{#1, #2}\(\R\)}}
\newcommand{\M}[1]{{\mathcal{M}_{#1}\(\R\)}}
\newcommand{\Mn}[1]{{\mathcal{M}_n\(#1\)}}
\newcommand{\MC}[1]{{\mathcal{M}_{#1}\(\C\)}}
\newcommand{\Id}[1]{{\mathrm{Id}_{#1}}}
\newcommand{\cond}[1]{{\mathrm{cond}\(#1\)}}
\newcommand{\partialx}[2]{\frac{\partial #1}{\partial #2}}
\newcommand{\partialxx}[2]{\frac{\partial^2 #1}{\partial #2^2}}
\newcommand{\uni}[2]{u_{{#1},{#2}}}

\setlength{\leftmargin}{-2cm}%
\setlength{\rightmargin}{2cm}%
\setlength{\textwidth}{15cm}

\begin{document}

\noindent
{\rule{\textwidth}{.2mm}}\\
ENSIIE
\hfill
Projet Math
\hfill
1A\\
{\rule{\textwidth}{.2mm}}

\section{EDP}

\begin{align*}
  \partialx{P}{t} & = rP - rS\partialx{P}{S} -
                    \frac{1}{2}\sigma^2S^2\partialxx{P}{S},
                    \quad\forall (t,s)\in[0,T]\times [0,L], \\
  P(T, s) & = \max(K-s, 0),\quad\forall s\in[0,L]\quad\mathrm{(CI)}, \\
  P(t, 0) & = Ke^{r\(t - T\)},\quad\forall t\in[0,T]\quad\mathrm{(CB1)}, \\
  P(t, L) & = 0,\quad\forall t\in[0,T]\quad\mathrm{(CB2)}.
\end{align*}

\section{Changement de variable}

On veut une condition initiale et non terminale. Posons
$u(t,x) = P(T- t, x)$. On obtient :

\begin{align*}
  \partialx{u}{t}(t, x) & = -\partialx{P}{t}(T - t, x), \\
                        & = -rP(T - t, x) + rx\partial{P}{x}(T - t, x) +
                          \frac{1}{2}\sigma^2x^2\partialxx{P}{x}(T - t, x), \\
                        & = -ru(t, x) + rx\partial{u}{x}(t, x) +
                          \frac{1}{2}\sigma^2x^2\partialxx{u}{x}(t, x).
\end{align*}

De m\^eme pour la condition initiale (CI) et les deux conditions aux
bord (CB1) et (CB2). On obtient au final :

\begin{align*}
  \partialx{u}{t} & = -ru + rx\partialx{u}{x} +
                                  \frac{1}{2}\sigma^2x^2\partialxx{u}{x},
                                  \quad\forall (t,x)\in[0,T]\times [0,L], \\
  u(0, x) & = \max(K - x, 0),\quad\forall x\in[0,L]\quad\mathrm{(CI)}, \\
  u(t, 0) & = Ke^{-rt},\quad\forall t\in[0,T]\quad\mathrm{(CB1)}, \\
  u(t, L) & = 0,\quad\forall t\in[0,T]\quad\mathrm{(CB2)}.
\end{align*}

\section{Discr\'etisation de $[0,T]$ et $[0, L]$}

Soient $N,\ M\in\N^*$. Posons $\Delta T=\frac{T}{N}$, $t_n =
n\Delta T\ \forall n\in\{0,\ldots,N\}$ et $\Delta x=\frac{L}{M+1}$, $x_i =
i\Delta x\ \forall i\in\{0,\ldots,M+1\}$

\section{Discr\'etisation de l'EDP}

On utilise un $\theta$-sch\'ema qui permet de prendre en compte la
m\'ethode des diff\'erences finies (MDF) explicite ($\theta = 0$),
implicite ($\theta = 1$) et de Crank-Nicholson ($\theta =
\frac{1}{2}$). $u_{n,i}$ sera l'approximation de $u\(t_n,x_i\)$ et
$\forall (n,i)\in\{1,\ldots,N - 1\}\times\{1,\ldots,M\}$ :

\begin{align}
  \nonumber
  \frac{\uni{n + 1}{i} - \uni{n}{i}}{\Delta T} =
  & -r\(\theta\uni{n + 1}{i} + (1 - \theta)\uni{n}{i}\) \\
  \nonumber
  & +r\frac{x_i}{\Delta x}\(\theta\(\uni{n + 1}{i + 1} - \uni{n +
    1}{i}\) + (1 - \theta)\(\uni{n}{i + 1} - \uni{n}{i}\)\) \\
  \label{eq:disc}
  & +\frac{1}{2}\sigma^2\frac{x_i^2}{\Delta x^2}\(\theta\(\uni{n +
    1}{i + 1} - 2\uni{n + 1}{i} + \uni{n + 1}{i - 1}\) + (1 - \theta)\(\uni{n}{i + 1} - 2\uni{n}{i} + \uni{n}{i - 1}\)\)
\end{align}

Condition initiale :
$$\uni{0}{i} = \max(K - x_i, 0),\quad \forall i\in\{1,\ldots,M + 1\}.$$

Conditions aux bords :
\begin{align*}
  \uni{n}{0}     & = e^{-rt_n},\quad \forall n\in\{1,\ldots,N\}, \\
  \uni{n}{M + 1} & = 0,\quad \forall n\in\{1,\ldots,N\}.
\end{align*}

On simplifie \eqref{eq:disc} en remarquant que $\frac{x_i}{\Delta x} =
i$ :

\begin{align}
  \nonumber
  \frac{\uni{n + 1}{i} - \uni{n}{i}}{\Delta T} =
  & -r\(\theta\uni{n + 1}{i} + (1 - \theta)\uni{n}{i}\) \\
  \nonumber
  & +ri\(\theta\(\uni{n + 1}{i + 1} - \uni{n +
    1}{i}\) + (1 - \theta)\(\uni{n}{i + 1} - \uni{n}{i}\)\) \\
  \label{eq:disc}
  & +\frac{1}{2}\sigma^2i^2\(\theta\(\uni{n +
    1}{i + 1} - 2\uni{n + 1}{i} + \uni{n + 1}{i - 1}\) + (1 - \theta)\(\uni{n}{i + 1} - 2\uni{n}{i} + \uni{n}{i - 1}\)\),
\end{align}

et on regroupe $\uni{n + 1}{i}$ \`a gauche et $\uni{n}{i}$ \`a droite
:

\begin{multline}
  (1+r\Delta T\theta)\uni{n+1}{i} - r\Delta T\theta i(\uni{n+1}{i+1} -
  \uni{n+1}{i}) - \\
  \frac{1}{2}\sigma^2\Delta T\theta i^2 (\uni{n+1}{i+1} -2\uni{n+1}{i}
  + \uni{n+1}{i - 1}) = \\
  (1-r\Delta T(1-\theta))\uni{n}{i} + r\Delta T(1-\theta) i(\uni{n}{i+1} -
  \uni{n}{i}) + \\
  \frac{1}{2}\sigma^2\Delta T(1-\theta) i^2 (\uni{n}{i+1} -2\uni{n}{i}
  + \uni{n}{i - 1}),
\end{multline}

\'equivalent \`a :

\begin{multline}
  \label{eq:disc2}
  -\frac{1}{2}\sigma^2\Delta T\theta i^2\uni{n+1}{i - 1} + (1+r\Delta
  T\theta +r\Delta T\theta i + \sigma^2\Delta T\theta i^2)\uni{n+1}{i}
  \\
  - (r\Delta T\theta i + \frac{1}{2}\sigma^2\Delta T\theta
  i^2\uni{n+1}{i + 1}) = \\
  \frac{1}{2}\sigma^2\Delta T(1-\theta) i^2\uni{n}{i - 1} + (1-r\Delta
  T(1-\theta) -r\Delta T(1-\theta) i - \sigma^2\Delta T(1-\theta)
  i^2)\uni{n}{i} \\
  + (r\Delta T(1-\theta) i + \frac{1}{2}\sigma^2\Delta T(1-\theta)
  i^2)\uni{n}{i + 1}.
\end{multline}

\section{Ecriture sous forme matricielle de \eqref{eq:disc2}}

Ne \textbf{pas} oublier les conditions initiale et aux bords. Pour
simplifier les expressions, posons :
\begin{align*}
  \alpha & = r\Delta T\theta i,\\
  \beta  & = \sigma^2\Delta T\theta i^2, \\
  \gamma & = r\Delta T(1 - \theta) i,\\
  \delta & = \sigma^2\Delta T(1-\theta) i^2. \\
\end{align*}

Alors \eqref{eq:disc2} est \'equivalent \`a

\begin{multline}
  \label{eq:disc3}
  -\frac{1}{2}\beta\uni{n+1}{i - 1} + (1+r\Delta
  T\theta + \alpha + \beta)\uni{n+1}{i}
  - \(\alpha + \frac{1}{2}\beta\)\uni{n+1}{i + 1} = \\
  \frac{1}{2}\delta\uni{n}{i - 1} + (1-r\Delta
  T(1-\theta) - \gamma - \delta)\uni{n}{i}
  + \(\gamma + \frac{1}{2}\delta\)\uni{n}{i + 1}.
\end{multline}

Conditions aux bords :
\begin{align*}
  \uni{n}{0}   & = Ke^{-rt_n} \mathrm{\ pour\ } i = 1 \\
  \uni{n}{M+1} &= 0 \mathrm{\ pour\ } i = M
\end{align*}

On pose le vecteur colonne $U_n=(\uni{n}{1},\ldots,\uni{n}{M})$. Avec \eqref{eq:disc3} et
(CB), la formulation \'equivalente est $AU_{n+1}=BU_n+F_n$, $A$ et $B$
deux matrices carr\'ees tridiagonales de taille $M$, et $F_n$ est un vecteur colonne
de taile $M$, avec :
\begin{align*}
  a_{i,i}  & = 1+r\Delta T\theta + \alpha + \beta,\ i=1\ldots M \\
  a_{i,i-1} & = -\frac{1}{2} \beta,\ i=2\ldots M\ \mathrm{sous-diag} \\
  a_{i+1,i} & = -\(\alpha + \frac{1}{2} \beta\),\ i=1\ldots M-1\ \mathrm{sur-diag} \\
  b_{i,i}  & = 1-r\Delta T(1-\theta) + \gamma + \delta,\ i=1\ldots M \\
  b_{i,i-1} & = \frac{1}{2} \delta,\ i=2\ldots M\ \mathrm{sous-diag} \\
  b_{i+1,i} & = \(\gamma + \frac{1}{2} \delta\),\ i=1\ldots M-1\
              \mathrm{sur-diag} \\
  f_1 & = \frac{1}{2}K\(\beta e^{-rt_{n+1}} + \delta e^{-rt_n}\) \\
      & = \frac{1}{2}Ke^{-rt_n}\(\beta e^{-r\Delta T} + \delta\) \\
  f_i & = 0 \ i=2,\ldots,M.
\end{align*}

\end{document}
